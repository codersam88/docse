\documentclass{article}
\begin{document}
	\section{INTRODUCTION}
		\subsection{Product overview}
		This project takes as  input Physical machines and their capacities Virtual machines and their 
		capacity requirements It computes the residual capacity in each physical machine after adding the 
		virtual machines. The physical machines are sorted in ascending order of their residual capacity. 
		The project provides the feature of consolidating the virtual machines in different physical machines into 
		minimum number of physical machines. Another feature provided by this project is to shutdown a physical system
		by migrating the virtual machines in that physical machine into other physical machines. This project uses
		greedy bin packing algorithm for this purpose. \section{SPECIFIC REQUIREMENTS}
		\subsection{External Interface Requirements}
			\subsubsection{User Interfaces}
			The graphical user interface will display the set of physical machines which are 'on' in a different color from those 				that are 'off'. It will also show the Ids of the virtual machines that are in each physical machine. 
			It will allow the user to
			\begin{itemize}
				\item add a virtual machine to a physical machine
				\item delete a virtual machine from a physical machine
				\item shut down a physical machine
			\end{itemize}

			\subsubsection{Hardware Interfaces}
			A visual display, speakers, a keyboard, and a mouse will be utilized to support graphical user interface.
			\subsubsection{Software Interfaces}
			It runs on every system that has a C compiler so it doesn't need any other specific software from the user side. It 				also depends on Graphics libraries in C 
			\subsubsection{Communication Protocols}
			NA
		\subsection{Software Product Features}
			\subsubsection {GetDetails}
			It reads the input from the input file parses the input and gets the following information
			\begin{itemize}
				\item The set of physical machines and their memory and CPU capabilities
				\item Memory and CPU requirements of virtual machines along with their Ids.
				\item Information about which virtual machine resides in which physical machine.
			\end{itemize}
			\subsubsection{Residual Capacity}
			It uses the information from the GetDetials and finds the Residual Capacity of every physical machine and finds which 				physical machine to be shut down.
			\subsubsection{Migrate}
			The victim physical machine which is chosen to shut down is processed to know if there are any virtual machines 			residing on that physical machine, if so they are migrated to other physical machines using the greedy bin packing 				algorithm. If there are no virtual machines in the victim physical machine then it is in 'off' state.
			\subsubsection{GUI}
			Graphical user Interface allow the users to
			\begin{itemize}
				\item add a virtual machine to a physical machine
				\item delete a virtual machine from a physical machine
				\item shut down the physical machine
			\end{itemize}
	\section{SOFTWARE SYSTEM ATTRIBUTES}
\end{document}
